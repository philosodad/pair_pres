
% $Header: /Users/paul/Classes/8220/Presentation/RCS/route-recovery.tex,v 1.1 2009/04/17 07:03:56 paul Exp $

\documentclass{beamer}

% This file is a solution template for:

% - Giving a talk on some subject.
% - The talk is between 15min and 45min long.
% - Style is ornate.



% Copyright 2004 by Till Tantau <tantau@users.sourceforge.net>.
%
% In principle, this file can be redistributed and/or modified under
% the terms of the GNU Public License, version 2.
%
% However, this file is supposed to be a template to be modified
% for your own needs. For this reason, if you use this file as a
% template and not specifically distribute it as part of a another
% package/program, I grant the extra permission to freely copy and
% modify this file as you see fit and even to delete this copyright
% notice. 


\mode<presentation>
{
  \usetheme{Frankfurt}
  % or ...
  
  \setbeamercovered{transparent}
  % or whatever (possibly just delete it)

  \usefonttheme[onlymath]{serif}
  
}

\usepackage{verbatim}
\usepackage{multirow} \usepackage{enumerate}
\usepackage{amsmath,enumerate} \usepackage{amsthm}
%\usepackage{algcompatible}
%\usepackage{algpseudocode}
\usepackage{algorithm}
%\usepackage{algorithmic}
%\usepackage{pstricks}
\usepackage{amssymb, latexsym}
\usepackage{xfrac}
\usepackage{mathtools}
\usepackage{graphicx}
%\usepackage[captionskip=5pt, nearskip=5pt, font=small]{subfig}
\DeclareGraphicsRule{*}{mps}{*}{}
%\usepackage{listings}

%pgfsettings
%\usepackage{pgf}
%\usepackage{tikz}
%\usetikzlibrary{decorations.pathmorphing} % LATEX and plain TEX when using Tik Z
%\usetikzlibrary{positioning}
%\tikzstyle{vx}=[draw,circle,fill=black!50,minimum size=2pt, inner sep=0pt, node distance=15mm]
%\tikzstyle{bup}=[decoration={bent, aspect=.3, amplitude=4}, decorate, ->, >=stealth]
%\tikzstyle{bdn}=[decoration={bent, aspect=.3, amplitude=-4}, decorate, ->, >=stealth]
%\tikzstyle{BUP}=[decoration={bent, aspect=.3, amplitude=8}, decorate, ->, >=stealth]
%\tikzstyle{BDN}=[decoration={bent, aspect=.3, amplitude=-8}, decorate, ->, >=stealth]
%\input{tikzamble.tex}
%\usepackage{algorithm, algorithmic}

\usepackage[english]{babel}
% or whatever

\usepackage[latin1]{inputenc}
% or whatever

\usepackage{times}
\usepackage[T1]{fontenc}
\usepackage{graphics}
% Or whatever. Note that the encoding and the font should match. If T1
% does not look nice, try deleting the line with the fontenc.

% le bib style
\bibliographystyle {IEEEtranS}


\title[] % (optional, use only with long paper titles)
{Pair Programming}

\subtitle
{} % (optional)

\author[] % (optional, use only with lots of authors)
{J. Paul Daigle }
% - Use the \inst{?} command only if the authors have different
%   affiliation.

\institute[] % (optional, but mostly needed)
{ DealerMatch
}

% - Use the \inst command only if there are several affiliations.
% - Keep it simple, no one is interested in your street address.

\date[] % (optional)
{2013-Jul-25}

\subject{}
% This is only inserted into the PDF information catalog. Can be left
% out. 



% If you have a file called "university-logo-filename.xxx", where xxx
% is a graphic format that can be processed by latex or pdflatex,
% resp., then you can add a logo as follows:

% \pgfdeclareimage[height=0.5cm]{university-logo}{university-logo-filename}
% \logo{\pgfuseimage{university-logo}}



% Delete this, if you do not want the table of contents to pop up at
% the beginning of each subsection:
\AtBeginSection[]
{
  \begin{frame}<beamer>
    \frametitle{Outline}
    \tableofcontents[currentsection,currentsubsection]
  \end{frame}
}


% If you wish to uncover everything in a step-wise fashion, uncomment
% the following command: 

%\beamerdefaultoverlayspecification{<+->}


\begin{document}

\begin{frame}
  \titlepage
\end{frame}

\begin{frame}
  \frametitle{Outline}
  \tableofcontents
  % You might wish to add the option [pausesections]
\end{frame}


\section{What is Pair Programming?}
\begin{frame}
  \frametitle{What is Pair Programming?}
  \begin{block}{Pair Programming}
  Pair Programming is the productivity practice of having two developers working at one computer.
  \end{block}
  \begin{block}{What's the Point?}
  Pairing improves overall team productivity by improving focus and fostering an energized working environment.\cite{6227110} 
  \end{block}
\end{frame}

\begin{frame}
  \frametitle{Why pair program?}
  \begin{enumerate}
  \item To improve communication 
  \item To improve the team's capability
  \item To improve productivity
  \item To improve product quality
  \end{enumerate}
\end{frame}


\section{Advantages of Pairing}
\begin{frame}
  \frametitle{Improving Focus}
  \begin{block}{Interuptions}
  In general, a pair is better able to return to work after an interuption than an individual\cite{6227110}. 
  \end{block} 
  \begin{block}{Distractions}
  In one study, pairs were found to spend twice as much time using the IDE and half as much time surfing the web or checking email\cite{854064}.
  \end{block}
  \begin{block}{Total Velocity}
  Studies that are largely critical of the claims of pair programming suggest that pairing either has little effect on velocity or increases overall velocity\cite{1553595}.
  \end{block}
\end{frame}

\begin{frame}
  \frametitle{Improving Results}
  \begin{block}{Reduction of Defects}
  When pairs work together well, they tend to produce code with fewer defects than single programmers\cite{Canfora20071317}.
  \end{block} 
  \begin{block}{Knowledge Sharing}
  Pairing is an effective way to get new programmers up to speed, especially when those programmers are novices to a skill or environment\cite{5315998}.
  \end{block}
\end{frame}

\begin{frame}[allowframebreaks]
  \frametitle{Bibliography}
  \bibliography{pairing}
\end{frame}

\end{document}

